\documentclass[journal,twocolumn]{IEEEtran}
\renewcommand{\baselinestretch}{1.45}

\usepackage{mathrsfs}
\usepackage{booktabs}
\usepackage{cite}

\usepackage{amssymb}
\usepackage{amsmath}
\usepackage{graphics}
\usepackage{color}
\usepackage{epsfig}
\usepackage{graphicx}
\usepackage{amsthm}
\usepackage{subfigure}
\allowdisplaybreaks[4]

\title{\LARGE Data-Driven Model Free Adaptive Sliding Mode Control For Multi DC Motors Speed Regulation}

\author{Tony Blaise Bimenyimana}

\begin{document}
\maketitle

% ---------------------------------------------------------------

\begin{abstract}
    This is Abstract
\end{abstract}

\section{Introduction}\label{section:1}

    In short time ago, the field of control systems experienced significant advancements, particularly in the domain of multi-agent systems. Distributed coordination of multi-agent systems has been utilized in range of practical applications, including satellite formation, autonomous, underwater vehicules,automated highway systems, and mobile robots. These applications highlight the importance and adaptability of multi-agent systems in addressing complex real world challenges.

    A critical aspect of multi-agent systems is robust speed regulation, effectively managing multi direct current (DC) motors is essential for acheiving complex machine movements. The complexity of such systems arises from the necessity to maintain synchronization and optimal performance despite parametric uncertainties and external disturbances. Conventional control methodologies, such us Proportional-Integral-Derivative (PID) control and Linear Quadratic Regulator (LQR), often fall short in these dynamic and unpredictable environments, highlighting the need for more adaptive and resilient approaches.

    Firstly, this paper aims to address these challenges by representing an innovative control strategy.Moreover, the proposed method synergizes the strengths of Model Free Adaptive Control (MFAC) and Sliding Mode Control (SMC) to achieve robust and efficient speed regulation in multi-agents systems. Secondly, By leveraging the adaptability of MFAC and the robustness of SMC, this approach seeks to enhance the performance and realibilty of multi DC motors speed in the presence of uncertainties and perturbations.

    Model Free Adaptive Control (MFAC) is a technique that eschews the need for an explicit mathematical model of the system, making it highly adaptive to real time changes and disturbances. This adaptability ensures optimal control performance in varying conditions. On the other hand, Sliding Mode Control (SMC) is well known for its robustness and ability to handle Nonlinearities and uncertainties by enforcing high frequency switching control law. SMC ensures that the system states are driven to and maintained within a specified sliding surface.

    Finally, the integration of these two methodologies facilitates a control strategy that is both adaptive and resilient. MFAC continiously adjusts the control parameters based on real-time system feedback, while SMC provides a robust framework to manage system uncertainties and external disturbances. This combination is particularly well-suited for the complex dynamics and interactions inherent in multi-agent systems, offering a comprehensive solution to the challenges of robust speed control.

    
\end{document}