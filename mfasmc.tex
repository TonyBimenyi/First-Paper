\documentclass[journal,twocolumn]{IEEEtran}
\renewcommand{\baselinestretch}{1.45}

\usepackage{mathrsfs}
\usepackage{booktabs}
\usepackage{cite}

\usepackage{amssymb}
\usepackage{amsmath}
\usepackage{graphics}
\usepackage{color}
\usepackage{epsfig}
\usepackage{graphicx}
\usepackage{amsthm}
\usepackage{subfigure}
\usepackage{amsmath}
\usepackage{amsfonts}
\usepackage{lettrine} % Package for dropped capital letters


\allowdisplaybreaks[4]

\title{\LARGE Data-Driven Model Free Adaptive Sliding Mode Control For Multi DC Motors Speed Regulation}

\author{Tony Blaise Bimenyimana}

\begin{document}
\maketitle

% ---------------------------------------------------------------

\begin{abstract}
    This is Abstract
\end{abstract}

\section{Introduction}\label{section:1}

    \lettrine{I}n short time ago, the field of control systems experienced significant advancements, particularly in the domain of multi-agent systems. Distributed coordination of multi-agent systems has been utilized in range of practical applications, including satellite formation, autonomous, underwater vehicules,automated highway systems, and mobile robots. These applications highlight the importance and adaptability of multi-agent systems in addressing complex real world challenges.

    A critical aspect of multi-agent systems is robust speed regulation, effectively managing multi direct current (DC) motors is essential for acheiving complex machine movements. The complexity of such systems arises from the necessity to maintain synchronization and optimal performance despite parametric uncertainties and external disturbances. Conventional control methodologies, such us Proportional-Integral-Derivative (PID) control and Linear Quadratic Regulator (LQR), often fall short in these dynamic and unpredictable environments, highlighting the need for more adaptive and resilient approaches.

    Firstly, this paper aims to address these challenges by representing an innovative control strategy. Moreover, the proposed method synergizes the strengths of Model Free Adaptive Control (MFAC) and Sliding Mode Control (SMC) to achieve robust and efficient speed regulation in multi-agents systems. Secondly, By leveraging the adaptability of MFAC and the robustness of SMC, this approach seeks to enhance the performance and realibilty of multi DC motors speed regulation in the presence of uncertainties and perturbations.

    Model Free Adaptive Control (MFAC) is a technique that eschews the need for an explicit mathematical model of the system, making it highly adaptive to real time changes and disturbances. This adaptability ensures optimal control performance in varying conditions. On the other hand, Sliding Mode Control (SMC) is well known for its robustness and ability to handle Nonlinearities and uncertainties by enforcing high frequency switching control law. SMC ensures that the system states are driven to and maintained within a specified sliding surface.

    Finally, the integration of these two methodologies facilitates a control strategy that is both adaptive and resilient. MFAC continiously adjusts the control parameters based on real-time system feedback, while SMC provides a robust framework to manage system uncertainties and external disturbances. This combination is particularly well-suited for the complex dynamics and interactions inherent in multi-agent systems, offering a comprehensive solution to the challenges of robust speed control.

    The following sections will outline the remaining content of this paper:
    section 2 provides-----. Section 3 ---------------------. Section 4 presents simulation results and performance analysis , demonstrating the efficacy of the proposed method under various operating conditions.
    Finally, Section 5 concludes the paper, summarizing the key findings potential avenues for future research, including the extension of the proposed approach to other types of multi-agent systems and exploring further enhancements to the control algorithms.

% -----------------------------------------------------------------------------


\section{Preliminaries and problem formulation}\label{section:2}
\subsection{Preliminaries}

The set of real numbers is denoted by \(\mathbb{R}\). For a given matrix \(A \in \mathbb{R}^{n \times n}\), \(\|A\|\) represents its matrix norm. The notation \(\text{diag}(\cdot)\) refers to a diagonal matrix, and \(I\) signifies an identity matrix of appropriate dimensions. In the context of multi-agent systems, graph theory serves as a powerful tool to model interaction topologies. We will now provide a brief introduction to directed graphs within algebraic graph theory. Let \(G = (V, E, A)\) be a weighted directed graph, where \(V = \{1, 2, \ldots, N\}\) represents the set of vertices, \(E \subseteq V \times V\) denotes the set of edges, and \(A\) is the adjacency matrix. Here, \(V\) also indexes the agents. If agent \(j\) can receive a message from agent \(i\), then \((i, j) \in E\), making \(j\) the child of \(i\) and \(i\) the parent of \(j\). The neighborhood of agent \(i\) is given by \(N_i = \{j \in V \mid (j, i) \in E\}\).

The weighted adjacency matrix \(A = (a_{i,j}) \in \mathbb{R}^{N \times N}\) is defined such that \(a_{i,i} = 0\), \(a_{i,j} = 1\) if \((j, i) \in E\); otherwise, \(a_{i,j} = 0\). The Laplacian matrix of \(G\) is defined as \(L = D - A\), where \(D = \text{diag}(d_1^{\text{in}}, d_2^{\text{in}}, \ldots, d_N^{\text{in}})\) and \(d_i^{\text{in}} = \sum_{j=1}^N a_{i,j}\) is called the in-degree of vertex \(i\). A graph is said to be strongly connected if there exists a path between any pair of vertices.

\subsection{Problem Formulation}


\end{document}