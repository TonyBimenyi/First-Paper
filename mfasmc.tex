\documentclass[journal,twocolumn]{IEEEtran}
\renewcommand{\baselinestretch}{1.45}

\usepackage{mathrsfs}
\usepackage{booktabs}
\usepackage{cite}

\usepackage{amssymb}
\usepackage{amsmath}
\usepackage{graphics}
\usepackage{color}
\usepackage{epsfig}
\usepackage{graphicx}
\usepackage{amsthm}
\usepackage{subfigure}
\allowdisplaybreaks[4]

\title{\LARGE Data-Driven Model Free Adaptive Sliding Mode Control For Multi DC Motors Speed Regulation}

\author{Tony Blaise Bimenyimana}

\begin{document}
\maketitle

% ---------------------------------------------------------------

\begin{abstract}
    This is Abstract
\end{abstract}

\section{Introduction}\label{section:1}

    In short time ago, the field of control systems experienced significant advancements, particularly in the domain of multi-agent systems. Distributed coordination of multi-agent systems has been utilized in range of practical applications, including satellite formation, autonomous, underwater vehicules,automated highway systems, and mobile robots. These applications highlight the importance and adaptability of multi-agent systems in addressing complex real world challenges.

    A critical aspect of multi-agent systems is robust speed regulation, effectively managing multi direct current (DC) motors is essential for acheiving complex machine movements. The complexity of such systems arises from the necessity to maintain synchronization and optimal performance despite parametric uncertainties and external disturbances. Conventional control methodologies often fall short in these dynamic and unpredictable environments, highlighting the need for more adaptive and resilient approaches.

    Firstly, this paper aims to address these challenges by representing an innovative control strategy. The proposed method synergizes the strengths of Model Free Adaptive Control (MFAC) and Sliding Mode Control (SMC) to achieve robust and efficient speed regulation in multi-agents systems. Secondly, By leveraging the adaptability of MFAC and the robustness of SMC, this approach seeks to enhance the performance and realibilty of multi DC motors speed 
\end{document}